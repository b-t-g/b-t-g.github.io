% Created 2018-04-26 Thu 19:51
% Intended LaTeX compiler: pdflatex
\documentclass[11pt]{article}
\usepackage[utf8]{inputenc}
\usepackage[T1]{fontenc}
\usepackage{graphicx}
\usepackage{grffile}
\usepackage{longtable}
\usepackage{wrapfig}
\usepackage{rotating}
\usepackage[normalem]{ulem}
\usepackage{amsmath}
\usepackage{textcomp}
\usepackage{amssymb}
\usepackage{capt-of}
\usepackage{hyperref}
\usepackage{amsfonts}
\usepackage{amssymb}
\usepackage{amsmath}
\usepackage{amscd}
\author{Brendan Good}
\date{2018-04-11 Wed 19:44}
\title{Understanding Natural Transformations}
\hypersetup{
 pdfauthor={Brendan Good},
 pdftitle={Understanding Natural Transformations},
 pdfkeywords={},
 pdfsubject={Natural transformations are critical in understanding category theory, here I'll explain them as I understand them},
 pdfcreator={Emacs 25.3.1 (Org mode 9.1.9)},
 pdflang={English}}
\begin{document}

\maketitle
Natural transformations act as the gatekeeper to the rest of category theory. Once you understand them, adjoint functors and even the (in)famous monads unveil themselves. Even
the "correct" notion of two categories being equivalent (unsurprisingly called an \href{https://en.wikipedia.org/wiki/Equivalence\_of\_categories}{equivalence of categories}) relies on natural transformations.

Despite this notion being critical, I've only really heard two explanations of them; both are one sentence long and both are unenlightening:
\begin{itemize}
\item Natural transformations are "coordinate free transformations" (what does that even mean)
\item A natural transformation is a "homomorphism of functors" (this is a bit too high-level to make sense to me "out of the box")
\end{itemize}
As is often the case for me, the best approach was to stare at the diagram and figure out the intuition for myself.

In this post, I'll talk about my intuition for natural transformations and, roughly, how I came to it. Although the understanding I ultimately arrived at is pretty much identical
to the "homomorphism of functors" viewpoint, this post unwinds that analogy a bit more.

\section*{What is a homomorphism, really?}
\label{sec:org4427e10}
At its core, a homomorphism is just a map which preserves the structure of its argument. For example, a group homomorphism is a map which sends groups to groups,
a "category homomorphism" (functor) sends categories to categories, etc. More generally, for an arbitrary category \(C\), if there exists a morphism \(f: X\to Y\), where \(X,Y\in\ ob(C)\)\footnote{Here I take \(ob(C)\) to mean the objects of the category \(C\).}
then we generally think of \(Y\) as having the same structure as \(X\) within \(C\). Even though that last point sounds like a needless abstraction, that insight will be useful later.

With this in mind, what is the structure of a functor? It's a strange question with a straightforward answer: you apply it. You give me an object \(X\in ob(C)\),
I'll give you an object \(F(X)\in ob(D)\). Similarly, you give me a morphism \(f: X\to Y\) where \(X,Y\in ob(C)\), I'll give you a morphism from \(F(f): F(X)\to F(Y)\). In short, we will show that a natural
transformation is the correct notion of one functor having the same structure as another.

\section*{Natural transformations}
\label{sec:org55e3d65}
\subsection*{Definition}
\label{sec:orga9692a1}
A natural transformation between two functors \(F,G:C\to D\) associates to each object \(X\in ob(C)\) a morphism \(\eta_X: F(X)\to G(X)\) such that the following diagram
commutes (which is a 90 degree rotation from the one \href{https://en.wikipedia.org/wiki/Natural\_transformation\#Definition}{listed on Wikipedia}).
\[\begin{CD}
F(X) @>\eta_X>> G(X) \\
@VF(f)VV        @VVG(f)V \\
F(Y) @>\eta_Y>> G(Y)
\end{CD} \]
Where \(f\in mor(X,Y)\)\footnote{Here I take \(mor(X,Y)\) to mean the morphisms between the objects \(X\) and \(Y\) in the category \(C\).}

\subsection*{The intuition}
\label{sec:org09fcd79}

How does this construction have those properties I talked about earlier? If I have two functors \(F\) and \(G\), I can apply it them to an object \(X\) and get two objects from \(D\), the top arrow says that
we require a morphism from \(\eta_X:F(X)\to G(X)\), which is to say that \(F(X)\) and \(G(X)\) have the same structure (whatever that means in our particular category).

That is only half the story, however. We also want to make sure that morphisms \(F(f)\) and \(G(f)\) have the same structure as well; but first, we have to define what that even means.

\section*{Structure of morphisms}
\label{sec:org64627bd}
Consider two morphisms \(f: X\to Y\) and \(g: Z\to W\) where \(X,Y,Z,W\in ob(C)\). What is a reasonable criterion to say that \(g\) has the same structure as \(f\)?
A morphism consists broadly of two things, a domain and a codomain. What does it mean to preserve the structure of the domain and codomain?

Recall that in a category \(C\), object \(Y\) "has the same structure" as \(X\) if there exists a morphism \(f: X\to Y\). With that in mind, we might say that a reasonable definition would be that two
morphisms have the same structure such that the domains and the codomains have the same structure (since the domain and codomain are just objects). In other words,
we want a diagram that looks like this (not necessarily commutative):

\[\begin{CD}
X    @>j>> Z \\
@VfVV      @VVgV \\
Y    @>k>> W
\end{CD}

\textbf{Diagram A}\]

I want to re-emphasize that, at this point, we do \textbf{not} require that this diagram be commutative (that is to say \(g\circ j\) does not necessarily equal \(k\circ f\)).

This looks fine at first glance, since we have that the domain and the codomain have the same structure, but consider the following (silly) example:

\[\begin{CD}
\mathbb{Z}              @>x\mapsto x>> \mathbb{Z} \\
@Vx\mapsto x\mod 2VV                   @VVx\mapsto x\mod 3V \\
\mathbb{Z}/ 2\mathbb{Z} @>0>>          \mathbb{Z}/ 3\mathbb{Z}
\end{CD} \]

If we compose the top arrow with the right arrow, we get the morphism \(x\mapsto x\mod 3\). However, when we compose the left arrow with the bottom arrow, we get the morphism \(x\mapsto 0\).
Because these resulting morphisms are different, it is possible to recover whether we mapped over via the domain or the codomain depending on what the resulting morphism is.
This indicates that the domain and codomain somehow have a different structure in this case.

So what we really want is to be able map from either the domain or the codomain and achieve the same morphism. Therefore, we say that \(g\) has the same structure as \(f\) if diagram A (above)
is commutative.

\subsection*{Intuition wrap-up}
\label{sec:org894c22d}
Therefore, a natural transformation is just a way of saying that one functor maps objects and morphisms in a way that has the same structure as another functor.
\section*{An odd example that helped it click for me}
\label{sec:org084df3b}
At a Haskell meetup, the organizer (\href{https://gbaz.github.io/}{Gershom Bazerman}) said "When people ask me for references on category theory, I say 'you always learn category theory from the second book you end up reading'";
I suspect that there may be an element of that here. Nonetheless, even though the example I'm about to provide is not a natural transformation, it helped me further understand
"functions that preserve the structure of other functions".


\subsection*{Chain complexes}
\label{sec:orgccc178d}

A chain complex \(C_\bullet\) is a collection of \(R-\text{modules}\)\footnote{If you aren't familiar with modules, replace all instances with "R-module" with "vector space" or "abelian group".} \(C_i\) along with module homomorphisms \(d_i: C_i\to C_{i-1}\) such that \(d_{i-1}\circ d_{i} = 0\)

\[\begin{CD}
... @>d_{n+2}>> C_{n+1} @>d_{n+1}>> C_n @>d_n>> C_{n-1} @>d_{n-1}>> ...\\
\end{CD}\]

In particular, the kernel of \(d_{i-1}\) is a submodule of the image of \(d_i\) (since \(d_{i-1}\circ d_{i} = 0\)), one of the things that we're interested in when we study chain complexes
 is the kernel of \(d_{i-1}\) (called cycles) and the image of \(d_i\) (called boundaries). In particular, we care about how much bigger the kernel is than the image; in other words,
we want to know the extent to which this diagram is not exact.

From now on, I'll simply refer to the morphisms \(d_i\) as simply \(d\).
\subsection*{Morphisms of chain complexes}
\label{sec:org2c187f7}

A morphism of chain complexes from \(C_\bullet \to D_\bullet\) is a collection of morphisms \(u_i: C_i\to D_i\) such that the following diagram commutes:

\[\begin{CD}
... @>d>> C_{n+1} @>d>> C_n @>d>> C_{n-1} @>d>> ...\\
@.        @Vu_{n+1}VV   @Vu_nVV     @Vu_{n-1}VV\\
... @>d>> D_{n+1} @>d>> D_n @>d>> D_{n-1} @>d>> ...
\end{CD}\]

It can be proven via \href{https://en.wikipedia.org/wiki/Five\_lemma\#Proof}{diagram chasing} that \(u\) sends cycles to cycles and boundaries to boundaries; which is to say, chain complex morphisms preserve precisely the structure that we're interested in.

Even though natural transformations may be intimidating at first, it is simply a way to say that two functors have the same structure; I hope this explanation has been helpful!
\end{document}
